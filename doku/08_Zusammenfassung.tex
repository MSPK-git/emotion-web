\documentclass[./dokumentation.tex]{subfiles}

\begin{document}
\chapter{Fazit}
In dieser Projektarbeit konnten wir nach ausführlicher Recherche eine Website zu den Emotionen ,,Gelassenheit'', ,,Nostalgie'' und ,,Stress/Unsicherheit/Wut'' entwickeln. Die theoretische Ausarbeitung liefert einen tiefen Einblick in die Funktionsweise der Emotionen Gelassenheit, Nostalgie und Stress/Wut, wodurch ein fundiertes Verständnis für das Projekt geschaffen wird. \\
Theoretisch konnten wir recherchieren, welche Möglichkeiten es gibt, verschiedene Emotionen auszulösen. Dabei waren insbesondere die visuellen Aspekte relevant, da verschiedene Farben eine unterschiedliche Wirkung auf einen Menschen haben können. Doch auch die Verwendung von Kontrasten, Typographien und audiovisuellen Inhalten sind hierbei relevant. \\

Insgesamt zeigt das Projekt, wie effektiv das Webdesign und die Farbgestaltung genutzt werden können, um bestimmte Emotionen gezielt anzusprechen und beim Nutzer eine entsprechende Reaktion hervorzurufen. Es verdeutlicht, dass die visuelle Gestaltung einer Webseite einen maßgeblichen Einfluss auf die emotionalen Reaktionen der Nutzer hat.



\end{document}


