\documentclass[./dokumentation.tex]{subfiles}
\begin{document}
\chapter{Einleitung}
\section{Einleitung}


THEMA: Konstruieren Sie eine Webseite, in welcher Sie zeigen, wie Sie bewusst mindestens drei Emotionen ansprechen. 

Webseiten als interaktives Medium bieten eine Vielzahl an Möglichkeiten, Emotionen hervorzurufen. Neben den Inhalten einer Webseite kann auch die Gestaltung genutzt werden, um zu beeinflussen, wie sich der Nutzer dabei fühlt. Beispiele dafür sind Geschichten, Bilder, Videos und Musik.
Neben dem Design einer Webseite, haben auch die Inhalte einen Einfluss auf die Emotionen der Nutzer. Daher ist es wichtig, das Publikum und dessen individuellen emotionalen Bedürfnisse zu verstehen. Nur so kann auch der entsprechende Inhalt erstellt werden, der diese Bedürfnisse erfüllen kann. Sind die Gefühle und Bedürfnisse der Nutzer erkannt und wurden verstanden, kann darauf mit einem benutzerzentrierten, emotionales Design reagiert werden.

Anhand von drei Beispielen wollen wir zeigen, wie die Emotionen Emotion1 Emotion2 und Emotion3 durch verschiedene, in dieser Arbeit näher erläuterte Faktoren, bei dem Besucher einer Webseite hervorgerufen werden können.



\end{document}
