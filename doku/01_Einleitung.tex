\documentclass[./dokumentation.tex]{subfiles}
\begin{document}
\chapter{Einleitung}

Das Internet ist als Medium in der vergangenen Dekade zunehmend in den Lebensmittelpunkt vieler Menschen gerückt. Durch die ständige Verfügbarkeit des Internets über Smartphone und Tablet sind Online-Inhalte permanent abrufbar und stehen unmittelbar bereit. Gleichzeitig hat sich die Webtechnologie stark weiterentwickelt. Webseiten sind inzwischen keine unidirektionalen Informationskanäle mehr, die statische, primär textliche Inhalte darstellen, sondern interaktive Medien, die multimedialen Content bereitstellen und zur Partizipation animieren.\\

Diese Webseiten als interaktive Medien bieten eine Vielzahl an Möglichkeiten, Emotionen hervorzurufen. Neben den Inhalten einer Webseite kann auch die Gestaltung genutzt werden, um zu beeinflussen, wie sich der Nutzer dabei fühlt. Beispiele dafür sind Geschichten, Bilder, Videos und Musik, aber auch die Farbgestaltung oder die Anwendung der Gestaltungsgesetze.\\
Neben dem Design einer Webseite haben auch die Inhalte einen Einfluss auf die Emotionen der Nutzer. Daher ist es wichtig, das Publikum und dessen individuellen emotionalen Bedürfnisse zu verstehen. Nur so kann auch der entsprechende Inhalt erstellt werden, der diese Bedürfnisse erfüllen kann. Sind die Gefühle und Bedürfnisse der Nutzer erkannt und wurden verstanden, kann darauf mit einem benutzerzentrierten, emotionales Design reagiert werden.\\

Anhand von drei Beispielen wollen wir zeigen, wie die Emotionen Wut, Nostalgie und Gelassenheit durch verschiedene, in dieser Arbeit näher erläuterte Faktoren, bei dem Besucher einer Webseite hervorgerufen werden können.\\




\end{document}
