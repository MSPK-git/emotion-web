\documentclass[./dokumentation.tex]{subfiles}

\begin{document}
\section{Verwendung vom Emotionen im Design}
Große Bilder (z.B.: Darstellungen von Personen vergrößern, nah an den Rahmen heranrücken), leuchtende Farben mit hoher Sättigung und hohe Kontraste auf Webseiten wirken erregender auf den Nutzer und können somit ein besonders wirkungsvoller Kanal sein, Aufmerksamkeit zu erhöhen und das Verhalten zu beeinflussen. Dabei erfolgt die Erregung größtenteils unbewusst und bietet somit für die Webdesigner einen besonders wirkungsvollen Kanal. Denn eine erhöhte Erregung verengt den Fokus der Aufmerksamkeit, sodass sich diese auf das richtet,  was die Stimulation (Erregung) verursacht (z.B.: Stoppschild) \cite{vanGorp2013}. 

\end{document}


