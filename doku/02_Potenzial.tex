\documentclass[./dokumentation.tex]{subfiles}

\begin{document}
\section{Potenzial von Emotionalem Design}
Ob der Nutzer eine Beziehung zu einer Webseite oder einem Produkt aufbauen kann, hängt davon ab, ob die Erfahrung, die er mit dieser gemacht hat, angenehm oder nützlich war. Trevor van Gorp geht in  \cite{vanGorp2013} sogar davon aus, dass der Grund für den Beziehungsaufbau zwischen Menschen vergleichbar ist. So soll man eher eine Beziehung aufbauen, wenn man den Besuch einer Webseite genießt und beim Verwenden dieser ein gutes Gefühl bekommt. Einen ähnlichen Effekt können auch attraktive Menschen haben. Ist eine Webseite dazu noch einfach zu bedienen und leicht zu verstehen, ist dies vergleichbar mit einem guten Gesprächspartner, zu dem man schneller eine Beziehung aufbaut. Wenn eine Webseite den Nutzer über einen längeren Zeitraum begleitet und zu seiner Zufriedenheit beträgt, ist auch dies ein Faktor für das Aufbauen einer tiefen Beziehung, ähnlich wie auch Menschen sich an langfristige Beziehungen untereinander binden, um ihre (emotionalen) Bedürfnisse zu erfüllen.  \cite{vanGorp2013}

\subsection{Ebenen der emotionales Verarbeitung}

Möchte man Nutzer emotional erreichen, ist es wichtig, die Ebenen der emotionalen Verarbeitung zu kennen und das User-Interface in diesen Bereichen entsprechend auszurichten. Dabei kann in drei verschiedene Ebenen unterschieden werden. 

%\subsection{Emotionen - Grundlagen}
\end{document}


